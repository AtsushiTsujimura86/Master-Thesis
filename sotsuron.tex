% !TEX encoding = UTF-8 Unicode
\documentclass[12pt]{jreport}
\usepackage{suthesis}
\usepackage[dvipdfm]{graphicx}
\usepackage{listings, jvlisting}
\usepackage{float}
\graphicspath{{figs/}}
\DeclareGraphicsExtensions{.pdf,.jpg,png,PNG}
\renewcommand{\lstlistingname}{ソースコード}
\lstset{
    frame=shadowbox,
    numbers=left,
    tabsize=2,
    breaklines=true,
    columns=fixed,
    basewidth=0.5em,
    framesep=0.2cm
}

% 表紙
\thesis{
\bf 2025年度 \\ 
%
% select one of the following
修士論文
}

\title{
\bf SN開発におけるデバッグシステムの開発 \\
}

\professor{
\begin{tabular}[t]{lr}
アサノ デービッド & 教授 \\
\end{tabular}
}

\date{`'
2025年1月30日提出
}

\author{
信州大学大学院 総合理工学研究科 工学専攻 情報数理・融合システム分野 \\ 
24W6047A\\
辻村篤志 \\
}

% 本文
\begin{document}
\maketitle
\maegaki

\begin{jabstract} %あらまし
あらましを書く。

\end{jabstract}

\maetsuke
\tableofcontents
\listoffigures
\listoftables

\hombun

%_/_/_/_/_/_第一章_/_/_/_/_/_
\chapter{序論} 
序論を書く。

%_/_/_/_/_/_第二章_/_/_/_/_/_
\chapter{背景技術} 
背景技術を書く。
\section{webページの実装}
\subsection{HTML}
\begin{par}
  HTMLとはHyper Text Markup Languageの略であり、webサイトのコンテンツの構造を作るためのマークアップ言語である。
\end{par}
\begin{par}
  HTMLのコードは大きく分けて、HTMLのメタ情報や外部リソースへのリンクなどを記述するhead要素と実際にwebページに表示する内容を記述するbody要素からなる。実際のHTMLのソースコードの例をソースコード\ref{html_src_1}に示す。\\
\end{par}



\begin{lstlisting}[caption=HTMLのソースコードの例,label=html_src_1]
  <!DOCTYPE html>
  <html>
    <head>
      <meta charset="UTF-8">
      <title>ページタイトル</title>
    </head>
    <body>
      <h1>h1タグのサンプル</h1>
      <input type="text" id="input_1">
    </body>
  </html>

  \end{lstlisting}



\vspace{1cm} % 1cmの垂直スペースを挿入  





%_/_/_/_/_/_第3章_/_/_/_/_/_
\chapter{システムの説明}

\section{システムの概要}
\begin{par}
システムの概要を書く。
\end{par}

%_/_/_/_/_/_第4章_/_/_/_/_/_
\chapter{操作例}
\clearpage

%_/_/_/_/_/_第5章_/_/_/_/_/_
\chapter{結論}
結論を書く。

% ========
% 参考文献
% ========
\bunken
\begin{thebibliography}{99}
\bibitem{citekey}参考文献の例


\end{thebibliography}

\end{document}
